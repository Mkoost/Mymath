\documentclass[12pt, a4paper]{article}

\usepackage[utf8]{inputenc}
\usepackage[russian]{babel}
\usepackage{geometry}
\usepackage{mathtools}
\usepackage{verbatim}
\usepackage{indentfirst}
\usepackage{caption}
\usepackage{subcaption}
\usepackage{import}
\usepackage{xifthen}
\usepackage{pdfpages}
\usepackage{transparent}
\usepackage{graphicx}
\usepackage{caption}
\usepackage{hyperref}



\usepackage[oglav,spisok,boldsect,eqwhole,figwhole,hyperref,hyperprint,remarks,greekit]{./style/fn2kursstyle}

\frenchspacing


\begin{document}
	

	
	\newpage
	
	% \section{Краткие сведения}
	% \subsection{Алгоритм Гаусса}
	% \subsection{Метод QR-разложения}
	% \section{Ход работы}
	% \subsection{Исходные данные}
	% \subsection{Результаты расчетов}
	% \subsection{Анализ результатов}
	
	\section-{Контрольные вопросы}
	\begin{enumerate}
		\item Каковы условия применимости метода Гаусса без выбора
		и с выбором ведущего элемента?
		
		\textit{\textbf{Ответ:}}
		
		Метод Гаусса без выбора ведущего элемента может быть применен над матрицей, обладающей не нулевыми угловыми минорами, поскольку каждый последующий элемент считается по формуле (\ref{gauss-1}), где $a_{i\,i}^{(i - 1)}$ обязан быть не нулевым. Предварительный выбор неизвестного, которое будет исключено, позволяет обойти данную проблему. Таким образом, условия применимости метода Гаусса с выбором ведущего элемента совпадают с критерием существования единственного решения системы.
		
		\item Вопрос
		
		\textit{\textbf{Ответ:}}
		
		\item В методе Гаусса с полным выбором ведущего элемента приходится не только переставлять уравнения, но и менять нумерацию неизвестных. Предложите алгоритм, позволяющий восстановить первоначальный порядок неизвестных.
		
		\textit{\textbf{Ответ:}}
		
		Чтобы восстановить изначальный порядок неизвестных, можно хранить их индексы в отдельном массиве, каждый раз меняя их местами вместе со столбцами. При окончании работы алгоритма необходимо отсортировать решение по переставленным индексам. 
		
		\item Вопрос
		
		\textit{\textbf{Ответ:}}
		
		\item Что такое число обусловленности и что оно характеризует? Имеется ли связь между обусловленностью и величиной определителя матрицы? Как влияет выбор нормы матрицы на оценку числа обусловленности?
		
		\textit{\textbf{Ответ:}}
		
		\item Вопрос
		
		\textit{\textbf{Ответ:}}
		
		\item Применимо ли понятие числа обусловленности к вырожденным матрицам?
		
		\textit{\textbf{Ответ:}}
		
		\item Вопрос
		
		\textit{\textbf{Ответ:}}
		
		\item Как можно объединить в одну процедуру прямой и обратный ход метода \mbox{Гаусса}? В чем достоинства и недостатки такого подхода?
		
		\textit{\textbf{Ответ:}}
		
		 Прямой ход метода \mbox{Гаусса} находит верхнюю треугольную матрицу $U$ для некоторой квадратной матрицы $A$. Следовательно, если использовать алгоритм над транспонированной матрицей $A^T$, то он найдет нижнюю треугольную матрицу $L$ для матрицы $A$. Таким образом, для полноценной реализации метода Гаусса достаточно дважды запустить процедуру прямого хода: сначала для матрицы $A$, затем для матрицы $U^T$. При таком подходе в разы упрощается реализация метода Гаусса, при этом происходит повторный проход по нулевым элементам матрицы $U^T$ и возникает необходимость в транспонировании~$U$, что может существенно увеличить время исполнения алгоритма .  
		
		\item Вопрос
		
		\textit{\textbf{Ответ:}}
		
	\end{enumerate}
	
\end{document}