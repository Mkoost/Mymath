\documentclass[12pt, a4paper]{article}

\usepackage[utf8]{inputenc}
\usepackage[russian]{babel}
\usepackage{geometry}
\usepackage{mathtools}
\usepackage{verbatim}
\usepackage{indentfirst}
\usepackage{caption}
\usepackage{subcaption}
\usepackage{import}
\usepackage{xifthen}
\usepackage{pdfpages}
\usepackage{array}
\usepackage{transparent}
\usepackage{graphicx}
\usepackage{caption}
\usepackage{hyperref}
\usepackage{float}

\newcommand{\norm}[1]{\lVert #1 \rVert}
\newcommand{\abs}[1]{\lvert #1 \rvert}
\usepackage[oglav,spisok,boldsect,eqwhole,figwhole,hyperref,hyperprint,remarks,greekit]{./style/fn2kursstyle}

\graphicspath{{./style/}{./figures/}}

\frenchspacing

\captionsetup[table]{justification=raggedleft, singlelinecheck=false}

\title{Численное решение краевых задач для одномерного уравнения
	теплопроводности}
\lab{2}
\author{М.\,А.~Каган}
\creator{И.\,А.~Яковлев}
\supervisor{}
\group{ФН2-62Б}
\date{2025}

\begin{document}
	\maketitle
	\tableofcontents
	
	\newpage
	
	
	
	\section-{Контрольные вопросы}
	
	\begin{enumerate}
		\item \textbf{Вопрос}
		\vspace*{0.2cm}
		
		\textit{\textbf{Ответ:}}
		
		\item \textbf{Какие из рассмотренных схем являются абсолютно устойчивыми? Какая из рассмотренных схем позволяет вести расчеты с более крупным шагом по времени?}
		\vspace*{0.2cm}
		
		\textit{\textbf{Ответ:}}
		
		\begin{enumerate}
			\item Пусть $y^I$, $y^{II}$ решение разностных задача с одинаковым оператором, соответствующим правым частям $\varphi^I$, $\varphi^{II}$ и граничным условиям $\nu^{I}$ и $\nu^{II}$.  Разностную схему называют абсолютно устойчивой, если существуют $M_1$ и $M_2$ большие нуля, не зависящие от шага сетки, что справедливо неравенство
			\[
			\norm{y^I - y^{II}} \le M_1 \norm{\varphi^I - \varphi^{II}}  + M_2\norm{\nu^{I} - \nu^{II}}
			\]
			вне зависимости от выбора соотношения шагов. Если при $M_1 = 0$ выполняется неравенство, то говорят об устойчивости по начальным условиям, а если $M_2$, то об устойчивости по правой части.
			
			Из рассмотренных схем, только смешанная разностная схема удовлетворяет данному условию.
			
			\item Для схем с безусловной аппроксимацией порядка $O(\tau^2 + h)$ можно вести расчет с б\'oльшим шагом по времени в сравнении с шагом $h$.   
		\end{enumerate}
		
		\item \textbf{Вопрос}
		\vspace*{0.2cm}
		
		\textit{\textbf{Ответ:}}
		
		\item \textbf{ Какие методы (способы) построения разностной аппроксимации граничных условий (2.5), (2.6) с порядком точности  $O(\tau^2 + h)$, $O(\tau^2 + h^2)$ и $O(\tau + h^2)$ вы знаете?}
		\vspace*{0.2cm}
		
		\textit{\textbf{Ответ:}}
		
		Схема вида: 
		$O(\tau^2 + h)$ и $O(\tau + h^2)$ при $\sigma = 1$ и $\sigma = 1/2$ соответственно
		
		
		\item \textbf{Вопрос}
		\vspace*{0.2cm}
		
		\textit{\textbf{Ответ:}}
		
		\item \textbf{Какие ограничения на $h$, $\tau$ и $\sigma$ накладывают условия устойчивости прогонки?}
		\vspace*{0.2cm}
		
		\textit{\textbf{Ответ:}}
		
		\begin{enumerate}
			Смешанная разностная сетка определяемая параметром $\sigma$ устойчива, если
			\[
			\sigma \ge \dfrac{1}{2} - \dfrac{c p h^2}{4 \tau \tilde{K}}, \;\;\;\; \tilde{K}=\max\limits_{0 \, \le x \, \le L}{K(x)}
			\]
			Для абсолютно устойчивых схем, в частности неявная, явная и симметричная, устойчивы при любых соотношениях шагов $\tau$ и $h$/
			
			Для $\sigma < 1/2$ устойчива при достаточно малом соотношении $\tau/h^2$, то такие схемы условно устойчивы.
		\end{enumerate}
		
		\item \textbf{Вопрос}
		\vspace*{0.2cm}
		
		\textit{\textbf{Ответ:}}
		
		\item \textbf{Для случая $K = K(u)$ предложите способы организации внутреннего итерационного процесса или алгоритмы, заменяющие его.}
		\vspace*{0.2cm}
		
		\textit{\textbf{Ответ:}}
		
	\end{enumerate}

	
	
	
\end{document}