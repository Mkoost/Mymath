\documentclass[12pt, a4paper]{article}

\usepackage[utf8]{inputenc}
\usepackage[russian]{babel}
\usepackage{geometry}
\usepackage{mathtools}
\usepackage{verbatim}
\usepackage{indentfirst}
\usepackage{caption}
\usepackage{subcaption}
\usepackage{import}
\usepackage{xifthen}
\usepackage{pdfpages}
\usepackage{transparent}
\usepackage{graphicx}
\usepackage{caption}
\usepackage{hyperref}
\usepackage{float}

\newcommand{\norm}[1]{\lVert #1 \rVert}
\newcommand{\abs}[1]{\lvert #1 \rvert}
\usepackage[oglav,spisok,boldsect,eqwhole,figwhole,hyperref,hyperprint,remarks,greekit]{./style/fn2kursstyle}

\graphicspath{{./style/}{./figures/}}

\frenchspacing

\title{ Методы численного решения обыкновенных дифференциальных уравнений}
\lab{1}
\author{М.\,А.~Каган}
\creator{И.\,А.~Яковлев}
\supervisor{}
\group{ФН2-51Б}
\date{2024}

\begin{document}
	\maketitle
	\tableofcontents
	
	\newpage
	
	
	
	\section-{Контрольные вопросы}
	
	\begin{enumerate}
		\item \textbf{Сформулируйте условия существования и единственности решения задачи Коши для обыкновенных дифференциальных уравнений. Выполнены ли они для вашего варианта задания?}
		\vspace*{0.2cm}
		
		\textit{\textbf{Ответ:}}
		
		Рассмотрим векторную функцию $u: I \subseteq \mathbb{R} \rightarrow \mathbb{R}^n$, где $t \in \mathbb{R}$. Рассмотрим задачу Коши:

		\[
		\begin{cases}
			u' = f(t, u) \\
			u(t_0) = u_0 \\
		\end{cases}
		\]
		\begin{enumerate}
			\item Пусть функция  $f(t, u)$ определена и непрерывна в прямоугольнике:
			\[
			D = \Bigl\{ (t, u): |t - t_0| \le a; |u_i - u_{0, i}| \le b \Bigr\}.
			\]
			Выберем $M > 0$, такую что $|f_i| < M$.
			\item Пусть функция $f(t, u)$ липшиц-непрерывна с постоянной $L$ по переменным $u_1, u_2, \ldots, u_n$:
			\[
			|f(t, u^{(1)}) - f(t, u^{(2)})| \le L\sum\limits^{n}_{i=1}|u^{(1)} - u^{(2)}|
			\]
		\end{enumerate}
		
		Тогда решение задачи Коши существует и единственно на участке
		\[
		|t - t_0| \le \min{a, b/M, 1/L}
		\]
		
		\item \textbf{Что такое фазовое пространство? Что называют фазовой траекторией? Что называют интегральной кривой?}
		\vspace*{0.2cm}
		
		\textit{\textbf{Ответ:}}
		
		\item \textbf{Каким порядком аппроксимации и точности обладают методы, рассмотренные в лабораторной работе?}
		\vspace*{0.2cm}
		
		\textit{\textbf{Ответ:}}
		
		\begin{enumerate}
			\item Метод Эйлера:
			\item Метод Рунге -- Кутты:
			\item Метод Адамса -- Башфорта:
			\item Метод <<предикор -- корректор>>:
		\end{enumerate}
		
		\item \textbf{Какие задачи называются жесткими? Какие методы предпочтительны для их решения? Какие из рассмотренных методов можно использовать для решения жестких задач?}
		\vspace*{0.2cm}
		
		\textit{\textbf{Ответ:}}
		
		\item \textbf{Как найти $\vec{y}_1$, $\vec{y}_2$, $\vec{y}_3$, чтобы реализовать алгоритм прогноза и коррекции (1.18)?}
		\vspace*{0.2cm}
		
		\textit{\textbf{Ответ:}}
		
		
		
		\item \textbf{ Какой из рассмотренных алгоритмов является менее трудоемким? Какой из рассмотренных алгоритмов позволяет достигнуть заданную точность, используя наибольший шаг интегрирования? Какие достоинства и недостатки рассмотренных алгоритмов вы можете указать?}
		\vspace*{0.2cm}
		
		\textit{\textbf{Ответ:}}
		
		\item \textbf{Какие алгоритмы, помимо правила Рунге, можно использовать для автоматического выбора шага?}
		\vspace*{0.2cm}
		
		\textit{\textbf{Ответ:}}
	
	\end{enumerate}

	
	
	
	
\end{document}