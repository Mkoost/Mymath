\documentclass[12pt, a4paper]{article}

\usepackage[utf8]{inputenc}
\usepackage[russian]{babel}
\usepackage{geometry}
\usepackage{mathtools}
\usepackage{verbatim}
\usepackage{indentfirst}
\usepackage{caption}
\usepackage{subcaption}
\usepackage{import}
\usepackage{xifthen}
\usepackage{pdfpages}
\usepackage{transparent}
\usepackage{graphicx}
\usepackage{caption}
\usepackage{hyperref}
\usepackage{float}

\newcommand{\norm}[1]{\lVert #1 \rVert}
\newcommand{\abs}[1]{\lvert #1 \rvert}
\newcommand{\prt}[2]{\frac{\partial#1}{\partial #2}}
\usepackage[oglav,spisok,boldsect,eqwhole,figwhole,hyperref,hyperprint,remarks,greekit]{./style/fn2kursstyle}

\graphicspath{{./style/}{./figures/}}

\frenchspacing

\title{ Численное решение краевых задач для двумерного уравнения Пуассона}
\lab{3}
\author{М.\,А.~Каган}
\creator{И.\,А.~Яковлев}
\supervisor{А. О. Гусев}
\group{ФН2-61Б}
\date{2025}

\begin{document}
	\maketitle
	\tableofcontents
	
	\newpage
	
	
	
	\section-{Контрольные вопросы}
	
	\begin{enumerate}
		\item \textbf{Оцените число действий, необходимое для перехода на следующий слой по времени методом переменных направлений.}
		\vspace*{0.2cm}
		
		\textit{\textbf{Ответ:}}
		
		% Л1 и Л2 --- разностные операторы второй производной по соответствующей переменной пространства 
		Запишем схему переменных направлений. Примем
		\begin{gather*}
			F(y) = \frac 2 \tau y + \Lambda_2 y + \phi, \quad F^k_{ij} = F(y^k_{ij}), \\
			\hat F(y) = \frac 2 \tau y + \Lambda_1 y + \phi, \quad \hat F^{k+ 1/2}_{ij} = \hat F(y^{k+ 1/2}_{ij}),
		\end{gather*}
		преобразовав уравнения с помощью введенных величин, получим
		\begin{gather*}
			\dfrac1{h_1^2} y^{k+1/2}_{i-1,j} - 2 \left(\dfrac1{h_1^2} + \dfrac1\tau\right) y^{k+1/2}_{ij} + \dfrac1{h_1^2} y^{k+1/2}_{i+1,j} = -F^k_{ij},\\
			u_{0,j} = \Omega_{0,j}, \quad u_{N_1,j} = \Omega_{N_1,j}, \quad j = 1, 2, \dots, N_2 -1,
		\end{gather*}
		где $\Omega_{i, j} = \xi(x_{i,1}, \, x_{2,j})$ --- значения искомой функции в граничых узлах области. Для вычисления $F^k_{ij}$ требуется порядка $3 N_1 N_2$ умножений. 2 и 3 строки представляет собой $N_2-1$ трехдиагональных СЛАУ размерности $N_1 - 1$. Для их решения требуется примерно $5 N_1 N_2$ операций. Такой же порядок операций получается и для остальных этапов:
		\begin{eqnarray*}
			\dfrac1{h_2^2} y^{k+1}_{i,j-1} - 2 \left(\dfrac1{h_2^2} + \dfrac1\tau\right) y^{k+1}_{ij} + \dfrac1{h_2^2} y^{k+1}_{i,j+1} = -\hat{F}^{k+1/2}_{ij},\\
			u_{i,0} = \Omega_{i,0}, \quad u_{i,N_2} = \Omega_{i,N_2}, \quad i = 1, 2, \dots, N_1 -1.
		\end{eqnarray*}
		Таким образом, для перехода на следующий слой по времени требуется порядка $16 N_1 N_2$ операций.
		
		\item \textbf{ Почему при увеличении числа измерений резко возрастает количество операций для решения неявных схем (по сравнению с одномерной схемой)?}
		\vspace*{0.2cm}
		 
		 При решении одномерной задачи аппроксимирующие уравнения зависят только от количества узлов на одной оси. Для $n$-мерных случаев количество неизвестных кратно количеству узлов на оси. Таким образом, если, например, СЛАУ решается методом Гаусса, то сложность алгоритма $O(N_1^3)$ для одномерного случая, а для n-мерного $O((N_1 N_2 \ldots N_n)^3)$ 
		\textit{\textbf{Ответ:}}
	\end{enumerate}
	
\end{document}