\documentclass[12pt, a4paper]{article}

\usepackage[utf8]{inputenc}
\usepackage[russian]{babel}
\usepackage{geometry}
\usepackage{mathtools}
\usepackage{verbatim}
\usepackage{indentfirst}
\usepackage{caption}
\usepackage{subcaption}
\usepackage{import}
\usepackage{xifthen}
\usepackage{pdfpages}
\usepackage{transparent}
\usepackage{graphicx}
\usepackage{caption}
\usepackage{hyperref}
\usepackage{float}

\newcommand{\norm}[1]{\lVert #1 \rVert}
\newcommand{\abs}[1]{\lvert #1 \rvert}
\newcommand{\prt}[2]{\frac{\partial#1}{\partial #2}}
\newcommand{\question}[1]{\item \textbf{#1}
	\vspace*{0.2cm}
	
	\textit{\textbf{Ответ:}}}
\usepackage[oglav,spisok,boldsect,eqwhole,figwhole,hyperref,hyperprint,remarks,greekit]{./style/fn2kursstyle}

\graphicspath{{./style/}{./figures/}}

\frenchspacing

\title{ Численное решение краевых задач для двумерного уравнения Пуассона}
\lab{3}
\author{М.\,А.~Каган}
\creator{И.\,А.~Яковлев}
\supervisor{А. О. Гусев}
\group{ФН2-61Б}
\date{2025}

\begin{document}
	\maketitle
	\tableofcontents
	
	\newpage
	
	
	
	\section-{Контрольные вопросы}
	
	\begin{enumerate}
		\item \textbf{Оцените число действий, необходимое для перехода на следующий слой по времени методом переменных направлений.}
		\vspace*{0.2cm}
		
		\textit{\textbf{Ответ:}}
		
		% Л1 и Л2 --- разностные операторы второй производной по соответствующей переменной пространства 
		Запишем схему переменных направлений. Примем
		\begin{gather*}
			F(y) = \frac 2 \tau y + \Lambda_2 y + \phi, \quad F^k_{ij} = F(y^k_{ij}), \\
			\hat F(y) = \frac 2 \tau y + \Lambda_1 y + \phi, \quad \hat F^{k+ 1/2}_{ij} = \hat F(y^{k+ 1/2}_{ij}),
		\end{gather*}
		преобразовав уравнения с помощью введенных величин, получим
		\begin{gather*}
			\dfrac1{h_1^2} y^{k+1/2}_{i-1,j} - 2 \left(\dfrac1{h_1^2} + \dfrac1\tau\right) y^{k+1/2}_{ij} + \dfrac1{h_1^2} y^{k+1/2}_{i+1,j} = -F^k_{ij},\\
			u_{0,j} = \Omega_{0,j}, \quad u_{N_1,j} = \Omega_{N_1,j}, \quad j = 1, 2, \dots, N_2 -1,
		\end{gather*}
		где $\Omega_{i, j} = \xi(x_{i,1}, \, x_{2,j})$ --- значения искомой функции в граничых узлах области. Для вычисления $F^k_{ij}$ требуется порядка $3 N_1 N_2$ умножений. 2 и 3 строки представляет собой $N_2-1$ трехдиагональных СЛАУ размерности $N_1 - 1$. Для их решения требуется примерно $5 N_1 N_2$ операций. Такой же порядок операций получается и для остальных этапов:
		\begin{eqnarray*}
			\dfrac1{h_2^2} y^{k+1}_{i,j-1} - 2 \left(\dfrac1{h_2^2} + \dfrac1\tau\right) y^{k+1}_{ij} + \dfrac1{h_2^2} y^{k+1}_{i,j+1} = -\hat{F}^{k+1/2}_{ij},\\
			u_{i,0} = \Omega_{i,0}, \quad u_{i,N_2} = \Omega_{i,N_2}, \quad i = 1, 2, \dots, N_1 -1.
		\end{eqnarray*}
		Таким образом, для перехода на следующий слой по времени требуется порядка $16 N_1 N_2$ операций.
		
		\item \textbf{ Почему при увеличении числа измерений резко возрастает количество операций для решения неявных схем (по сравнению с одномерной схемой)?}
		\vspace*{0.2cm}
		
		\textit{\textbf{Ответ:}} 
		
		 При решении одномерной задачи аппроксимирующие уравнения зависят только от количества узлов на одной оси. При увеличении размерности общее количество узлов увеличится кратно их количеству на добавляемых осях, а соотвественно и количество решаемых уравнений. Таким образом, если, например, СЛАУ решается методом Гаусса, то сложность алгоритма $O(N_1^3)$ для одномерного случая, а для n-мерного $O((N_1 N_2 \ldots N_n)^3)$ 
		
		\item \textbf{Можно ли использовать метод переменных направлений в областях произвольной формы?}
		\vspace*{0.2cm}
		
		\textit{\textbf{Ответ:}}
		
		Напрямую применять схему Писмена-Рекфорда в областях проивзольной формы нельзя. Регулярная декомпозиция по координатным направлениями становится некорректной или невозможной: строки или столбцы сетки могут выходить за границы.
		
		Однако имеются обходные пути. Первый состоит в том, чтобы построить сетку в квадрате и <<вырезать>> необходимую форму, используя граничные условия. При этом диапазон изменения индекса $i$ будет зависеть от значения $j$ и наоборот. Вторым способом может быть переход к криволинейным координатам, если возможно отобразить данную произвольную область в квадрат и адаптировать схему.
		
		\item \textbf{Можно ли использовать метод переменных направлений в областях произвольной формы?}
		\vspace*{0.2cm}
		
		\textit{\textbf{Ответ:}}
		
		Продольно-поперечная схема на задачи с $p \ge 3$ непосредственно не обобщается вследствие возникающих несимметричности и условной устойчивости. Имевшаяся в двумерном случае симметричность давала равные (по модулю) ошибки с разными знаками на двух последовательных шагах, компенсировавшие друг друга.
		
		Однако в таком случае можно использовать локально-одномерную схему с использованием промежуточных слоев. Эта схема имеет лишь суммарную аппроксимацию, а на промежуточных слоях она не аппроксимирует исходное дифференциальное уравнение. Однако ошибки аппроксимации при суммировании гасят друг друга, так что решение на <<целом>> слое оказывается приближением точного.
		
		Рассмотрим уравнение
		\[
		u_t = \sum_{i = 1}^p u_{x_i x_i} + f.
		\]
		
		Аппроксимируем это уравнение, используя симметричную неявную схему
		\[
		y_t = \sum_{i = 1}^p \Lambda_i y^{(0,5)} + \varphi,
		\]
		($\Lambda_i$ --- разностная вторая производная по координате $x_i$).
		
		Наряду с исходной схемой построим локально-одномерную схему. Для этого между слоями $t$ и $\hat t$ введем $p + 1$ промежуточных слоев с шагами $\tau / p$ между ними. Первый слой соответствует моменту времени $t$, последний с номером $p+1$ --- моменту времени $\hat t$. На каждом таком слое с номером $\alpha$ суммарный оператор в правой части заменим оператором $\Lambda_\alpha$. Обозначим решение на промежуточных шагах через $w_\alpha$, $\alpha = 1, 2, \dots, p$. Тогда $w_\alpha$ является решением следующей разностной задачи:
		\begin{gather}
			\dfrac1\tau (\hat w_\alpha - w_\alpha) = \dfrac12 \Lambda_\alpha(\hat w_\alpha + w_\alpha) + \varphi_\alpha, \quad \alpha = 1, 2, \dots, p; \\
			w_1 = y, w_2 = \hat w_1, \dots, w_p = \hat w_{p-1}, \hat w_p = \hat y.
		\end{gather}
		
		Очевидно, что для любого $p$ соответствующее уравнение является одномерным, решаемым методом обычной прогонки. Остальные независимые переменные участвуют в нем только в качестве параметров. Поэтому и схема называется локально-одномерной.
		
		\item\textbf{Можно ли использовать метод переменных направлений на неравномерных сетках?}
		\vspace*{0.2cm}
		
		\textit{\textbf{Ответ:}}
		
		Нельзя, так как разделение направлений возможно только тогда, когда сетка по каждому направлению независима от других. На неравномерной сетке это условие нарушается.
	\end{enumerate}
	
	\section-{Дополнительные вопросы}
	\begin{enumerate}
		
	\question{Влияет ли начальное условие на метод счета на установления?}
	
	Для простоты рассмотрим уравнение Пуассона с граничными условиями 1-го рода:
	\begin{gather*}
		\Delta u(x, y) = -f(x, y), \quad (x, y) \in V \\
		u(x, y) = g(x, y),  (x, y), \quad \in \Gamma = \partial V
	\end{gather*}
	
	Рассмотрим дополнительную задачу:
	
	\begin{gather*}
		 v_t(x, y, t) = \Delta v(x, y, t) + f(x, y, t), \quad (x, y) \in V, \; t>0 \\
		v(x, y, t) = g(x, y, t), \quad (x, y)  \in \Gamma, \; t>0 \\
		v(x, y, 0) = v_0(x, y) 
	\end{gather*}
	
	Введем замену $w = u - v$. Преобразовав предыдущее, выражение получаем:
	
	\begin{gather*}
		w_t(x, y, t) = \Delta w(x, y, t), \quad (x, y) \in V, \; t>0 \\
		w(x, y, t) = 0, \quad (x, y)  \in \Gamma, \; t>0 \\
		w(x, y, 0) = w_0(x, y) 
	\end{gather*}
	
	Данное уравнение имеет решение следующего вида:
	\[
	w(x, y, t) = \sum\limits_{n = 1}^{\infty}\sum\limits_{m = 1}^{\infty}e^{-\lambda_{nm}^2 t}w_{nm}(x, y) \xrightarrow{t \rightarrow +\infty} e^{-\min\limits_{n, m}\lambda_{nm}^2 t}w_{nm}(x, y) \rightarrow 0
	\]
	То есть решение на бесконечности стремится к 0. Из этого следует $v \xrightarrow{t \rightarrow +\infty} u$, а поскольку при любых ограниченных собственных функций, которые зависят от произвольных начальных условий, решение стремится к искомой функции, следует, что метод счета на установления не зависит от выбора начальных условий. 
	
	\question{Почему решается нестационарная задача, а не исходная?}
	
	Поскольку если напрямую аппроксимировать двумерную задачу Пуассона возникает сложная матрица порядка $N^2$. Решение методом гаусса слишком сложное $O(N^{6})$. Можно решать ее итерационными методами. Таким образом, решение задачи сводится к поиску предельного значения, как и в случае метода счета на установление. 
	
	При решении задачи с помощью метода переменных направлений каждый переход на следующий временной слой требует порядка $O(N^2)$  операций. Таким образом, необходимо порядка $O(N^3)$ операций для выхода на стационарное решение, что намного быстрее, решения задачи методом Гаусса.
\end{enumerate}
	
	
\end{document}