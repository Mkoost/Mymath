\documentclass[12pt, a4paper]{article}

\usepackage[utf8]{inputenc}
\usepackage[russian]{babel}
\usepackage{geometry}
\usepackage{mathtools}
\usepackage{verbatim}
\usepackage{indentfirst}
\usepackage{caption}
\usepackage{subcaption}
\usepackage{import}
\usepackage{xifthen}
\usepackage{pdfpages}
\usepackage{transparent}
\usepackage{graphicx}
\usepackage{caption}
\usepackage{hyperref}
\usepackage{float}

\newcommand{\norm}[1]{\lVert #1 \rVert}
\newcommand{\abs}[1]{\lvert #1 \rvert}
\newcommand{\prt}[2]{\frac{\partial#1}{\partial #2}}
\usepackage[oglav,spisok,boldsect,eqwhole,figwhole,hyperref,hyperprint,remarks,greekit]{./style/fn2kursstyle}

\graphicspath{{./style/}{./figures/}}

\frenchspacing

\title{ Численное решение краевых задач для одномерного волнового уравнения}
\lab{3}
\author{М.\,А.~Каган}
\creator{И.\,А.~Яковлев}
\supervisor{А. О. Гусев}
\group{ФН2-61Б}
\date{2025}

\begin{document}
	\maketitle
	\tableofcontents
	
	\newpage
	
	
	
	\section-{Контрольные вопросы}
	
	\begin{enumerate}
		\item \textbf{Предложите разностные схемы, отличные от схемы «крест»,
			для численного решения волнового уравнения с граничными условиями.}
		\vspace*{0.2cm}
		
		\textit{\textbf{Ответ:}}
		
		\begin{enumerate}
			\item Неявная схема:
			\[
			\dfrac{\hat{y} - 2y + \check{y}}{\tau^2} = \dfrac{\hat y_{+1} -2 \hat y + \hat y_{-1}}{h ^ 2}
			\]
			\item Экстраполяционная схема:
			\[
			\dfrac{\hat{y} - 2y + \check{y}}{\tau^2} = \dfrac{\check y_{+1} -2 \check y + \check y_{-1}}{h ^ 2}
			\]
			\item Экстраполяционная схема:
			\[
			\dfrac{\hat{y} - 2y + \check{y}}{\tau^2} = \dfrac{\check y_{+1} -2 \check y + \check y_{-1}}{h ^ 2}
			\]
			
		\end{enumerate}
		
		\item \textbf{Предложите способ контроля точности полученного решения.}
		\vspace*{0.2cm}
		
		\textit{\textbf{Ответ:}}		
		
		Зная порядок аппроксимации схемы, по правилу Рунге каждые два шага можно оценивать погрешность решения и при, необходимости, уменьшать шаг интегрирования по времени.  
		
	\end{enumerate}
	
\end{document}