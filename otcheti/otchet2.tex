\documentclass[12pt, a4paper]{article}

\usepackage[utf8]{inputenc}
\usepackage[russian]{babel}
\usepackage{geometry}
\usepackage{mathtools}
\usepackage{verbatim}
\usepackage{indentfirst}
\usepackage{caption}
\usepackage{subcaption}
\usepackage{import}
\usepackage{xifthen}
\usepackage{pdfpages}
\usepackage{transparent}
\usepackage{graphicx}
\usepackage{caption}
\usepackage{hyperref}
\usepackage{float}

\newcommand{\norm}[1]{\lVert #1 \rVert}
\newcommand{\abs}[1]{\lvert #1 \rvert}
\usepackage[oglav,spisok,boldsect,eqwhole,figwhole,hyperref,hyperprint,remarks,greekit]{./style/fn2kursstyle}

\graphicspath{{./style/}{./figures/}}

\frenchspacing

\title{Итерационные метды решения систем линейных алгебраических уравнений}
\lab{1}
\author{М.\,А.~Каган}
\creator{И.\,А.~Яковлев}
\supervisor{}
\group{ФН2-51Б}
\date{2024}

\begin{document}
	\maketitle
	\tableofcontents
	
	\newpage
	
	\section{Исходные данные}
	
	Система линейных уравнений №1:
	\[
	\left \{ \begin{array}{ccccccccc}
		0.2910 x_1  & +   &   1.8100 x_2 & +   &     9.3110 x_3 & +    &    9.1100 x_4 & = & 4.2280\\
		1.4500  x_1 & + &     8.5790 x_2 & +  &     44.1950 x_3  & +  &    42.9950 x_4  & = & 20.4290\\
		-0.2900 x_1  & - &     1.7980 x_2 & - &      9.2500 x_3   & - &    9.0500 x_4 & = & -4.2080  \\
		0.0000  x_1  & + &     0.0820 x_2  & +  &     0.4100 x_3 & +  &      0.4500 x_4 & = & 0.1220
	\end{array}	\right.
	\]
	 
	Система линейных уравнений №2:
	 \[
	 \left \{ \begin{array}{ccccccccc}
	 -106.4000 x_1 & - & 7.0000 x_2 & - & 4.9900 x_3 & + & 0.2600 x_4 & = & 1040.8100\\
	3.6100 x_1 & + & 22.2000 x_2 & - & 8.5900 x_3 & - & 8.9200 x_4 & = & 615.4100 \\
	 2.2800 x_1 & + & 7.7500 x_2 & + & 52.2000 x_3 & + & 9.6500 x_4 & = & 427.5400 \\
	 -9.0000 x_1 & + & 5.8100 x_2 & - & 0.0900 x_3 & + & 136.8000 x_4 & = & -265.3500
	\end{array}	\right.
	 \]  
	\section{Краткие сведения}
	Пусть $A$ --- невырожденная матрица $n \times n$, $b$ --- ненулевой n-мерный вектор. Необходимо найти такой n-мерный вектор $x$, чтобы он удовлетворял уравнению
	\begin{equation}
		\label{eq}
		A x = b.
	\end{equation}
  
	\subsection{Метод простой итерации}
	
	\subsection{Метод Якоби}
			
	\subsection{Метод Зейделя}

	\subsection{Метод релаксации}	
	% \section{Ход работы}
	% \subsection{Исходные данные}
	% \subsection{Результаты расчетов}
	% \subsection{Анализ результатов}
	
	\section-{Контрольные вопросы}
	\begin{enumerate}
		\item \textbf{Почему условие $\norm{C} < 1$ гарантирует сходимость итерационных методов?}
		
		\vspace*{0.2cm}
		\textit{\textbf{Ответ:}}
	
Условие $\norm{C} < 1$ связано с тем, что для нахождения единственного решения (единственной неподвижной точки) оператору $C$ необходимо быть сжимающим, а из определения сжимающего оператора следует, что его норма должна быть строго меньше единицы.

		\textbf{\item оп оп}
		
		\vspace*{0.2cm}
		\textit{\textbf{Ответ:}}

		\textbf{\item На примере системы из двух уравнений с двумя неизвестными дайте геометрическую интерпретацию метода Якоби, метода Зейделя, метода релаксации.}
		
		\vspace*{0.2cm}
		\textit{\textbf{Ответ:}}

Рассмотрим метод Якоби. В этом случае итерационный процесс организуется следующим образом:
\begin{equation*}
        	\begin{cases}
        		a_{11} x^{k+1}_1 + a_{12} x^k_2 = f_1,\\
        		a_{21} x^{k+1}_1 + a_{22} x^k_2 = f_2.\\
        	\end{cases}
\end{equation*}
Каждое из уравнений задает некоторую прямую, точное решение $\hat{x}$ лежит на их пересечении. Приведем картинку с поэтапным поиском приближений:

		\textbf{\item оп оп}
		
		\vspace*{0.2cm}
		\textit{\textbf{Ответ:}}

		\textbf{\item Выпишите матрицу $C$ для методов Зейделя и релаксации.}
		
		\vspace*{0.2cm}
		\textit{\textbf{Ответ:}}
		
		В матричном виде метод Зейделя задается как:
\begin{equation*}		
(D+ L) \left( x^{k+1} - x^k \right)+ Ax^k = b.
\end{equation*}
Необходимо получить в левой части $x^{k+1}$, а в правой --- свободный член и $x^k$ умноженный на некоторый матричный коэффициент. Собрав множители при $x^k$ и перенеся его в правую часть, получим:
\begin{equation*}		
(D+ L)x^{k+1}  = (D+ L- A)x^{k} + b.
\end{equation*}
Домножим обе части на $(D+\omega L)^{-1}$:
\begin{equation*}
x^{k+1}  = (D+\omega L)^{-1}(D+L-A)x^{k} + (D+ L)^{-1}b,
\end{equation*}		
откуда $C = (D+L)^{-1}(D+L-A) = (D+L)^{-1}(-U) = - (D+L)^{-1}U$.

		\textbf{\item оп оп}
		
		\vspace*{0.2cm}
		\textit{\textbf{Ответ:}}

		\textbf{\item Какие еще критерии окончания итерационного процесса можно предложить?}
		
		\vspace*{0.2cm}
		\textit{\textbf{Ответ:}}

Можно воспользоваться следующими критериями останова:
\begin{equation*}
\begin{aligned}
&\norm{x^{k+1} - x^k} \leq \varepsilon \norm{x^k} + {\varepsilon }_0, 
&\norm{A x^{k + 1}} \leq \varepsilon .
\end{aligned}
\end{equation*}
Однако у них есть существенный недостаток: они не могут гарантировать условия $\norm{x^k - \hat{x}} \leq \varepsilon$, то есть сходимости к точному решению.
		
	\end{enumerate}
\end{document}