\documentclass[12pt, a4paper]{article}

\usepackage[utf8]{inputenc}
\usepackage[russian]{babel}
\usepackage{geometry}
\usepackage{mathtools}
\usepackage{verbatim}
\usepackage{indentfirst}
\usepackage{caption}
\usepackage{subcaption}
\usepackage{import}
\usepackage{xifthen}
\usepackage{pdfpages}
\usepackage{transparent}
\usepackage{graphicx}
\usepackage{caption}
\usepackage{hyperref}
\usepackage{float}

\newcommand{\norm}[1]{\lVert #1 \rVert}
\newcommand{\abs}[1]{\lvert #1 \rvert}
\usepackage[oglav,spisok,boldsect,eqwhole,figwhole,hyperref,hyperprint,remarks,greekit]{./style/fn2kursstyle}

\graphicspath{{./style/}{./figures/}}

\frenchspacing

\title{Итерационные метды решения систем линейных алгебраических уравнений}
\lab{2}
\author{М.\,А.~Каган}
\creator{И.\,А.~Яковлев}
\supervisor{}
\group{ФН2-51Б}
\date{2024}

\begin{document}
	\maketitle
	\tableofcontents
	
	\newpage
	

	
	\section-{Контрольные вопросы}
	
	\begin{enumerate}
	\item \textbf{-}
	\vspace*{0.2cm}
	
	\textit{\textbf{Ответ:}}
	
	\item \textbf{Докажите, что ортогональное преобразование подобия сохраняет симметрию матрицы.}
	\vspace*{0.2cm}
	
	\textit{\textbf{Ответ:}}
	
	Пусть $A$ --- симметричная матрица, т.е. $A = A^\text{T}$; $Q$ --- ортогональная матрица. По свойству ортогональных матриц:  $Q^{-1} = Q^\text{T}$. Рассмотрим ортогональное подобное преобразование :
	\begin{gather}
		B = Q A Q^{-1} = Q A Q^\text{T} \notag \\
		B^\text{T} = (Q A Q^\text{T})^\text{T} = (A Q^\text{T})^\text{T} Q^\text{T} = Q A^\text{T} Q^\text{T} = Q A Q^\text{T} = B \notag
	\end{gather}
	
	\item \textbf{-}
	\vspace*{0.2cm}
	
	\textit{\textbf{Ответ:}}
	
	\item \textbf{Почему на практике матрицу $A$ подобными преобразованиями вращения приводят только к форме Хессенберга, но не к треугольному виду?}
	
	\vspace*{0.2cm}
	
	\textit{\textbf{Ответ:}}
	\begin{enumerate}
		\item Могут <<потеряться>> комплексные собственные числа, поскольку матрицы поворота состоят из действительных чисел;
		\item QR-алгоритм сохраняет форму Хессенберга;
	\end{enumerate}

	
	\item \textbf{-}
	\vspace*{0.2cm}
	
	\textit{\textbf{Ответ:}}
	
	\item \textbf{Сойдется ли алгоритм обратных итераций, если в качестве начального приближения взять собственный вектор, соответствующий другому собственному значению? Что будет в этой ситуации в методе обратной итерации, использующем отношение Рэлея?}
	
	\vspace*{0.2cm}
	
	\textit{\textbf{Ответ:}}
	
	
	\item \textbf{-}
	\vspace*{0.2cm}
	
	\textit{\textbf{Ответ:}}
	
	\item \textbf{ Предложите возможные варианты условий перехода к алгоритму со сдвигами. Предложите алгоритм выбора величины сдвига.}
	\vspace*{0.2cm}
	
	\textit{\textbf{Ответ:}}
	
	\item \textbf{-}
	\vspace*{0.2cm}
	
	\textit{\textbf{Ответ:}}
	
	\item \textbf{Приведите примеры использования собственных чисел и собственных векторов в численных методах}
	\vspace*{0.2cm}
	
	\textit{\textbf{Ответ:}}
	\begin{enumerate}
		\item Критерий сходимости стационарных итерационных методов решения СЛАУ (спектральный радиус меньше единицы); 
		\item Оптимальные итерационные параметры для методов решения СЛАУ нередко требуют знания собственных значений матрицы (оптимальный параметр для метода простой итерации для симметричной, положительно определенной матрицы, для метода Ричардсона с Чебышевскими параметрами и т.д.);
		\item Оценка обусловленности матрицы;
		\item Решение системы линейных ОДУ с постоянными коэффициентами; 
		\item Привидение к канонической виду квадратичной формы. 
	\end{enumerate}
	
	\end{enumerate}
\end{document}