\documentclass[12pt, a4paper]{article}

\usepackage[utf8]{inputenc}
\usepackage[russian]{babel}
\usepackage{geometry}
\usepackage{mathtools}
\usepackage{verbatim}
\usepackage{indentfirst}
\usepackage{caption}
\usepackage{subcaption}
\usepackage{import}
\usepackage{xifthen}
\usepackage{pdfpages}
\usepackage{transparent}
\usepackage{graphicx}
\usepackage{caption}
\usepackage{hyperref}
\usepackage{float}

\newcommand{\norm}[1]{\lVert #1 \rVert}
\newcommand{\abs}[1]{\lvert #1 \rvert}
\usepackage[oglav,spisok,boldsect,eqwhole,figwhole,hyperref,hyperprint,remarks,greekit]{./style/fn2kursstyle}

\graphicspath{{./style/}{./figures/}}

\frenchspacing

\title{Итерационные метды решения систем линейных алгебраических уравнений}
\lab{2}
\author{М.\,А.~Каган}
\creator{И.\,А.~Яковлев}
\supervisor{}
\group{ФН2-51Б}
\date{2024}

\begin{document}
	\maketitle
	\tableofcontents
	
	\newpage
	

	
	\section-{Контрольные вопросы}
	
	\begin{enumerate}
	\item \textbf{-}
	\vspace*{0.2cm}
	
	\textit{\textbf{Ответ:}}
	
	\item \textbf{Докажите, что ортогональное преобразование подобия сохраняет симметрию матрицы.}
	\vspace*{0.2cm}
	
	\textit{\textbf{Ответ:}}
	
	Пусть $A$ --- симметричная матрица, т.е. $A = A^\text{T}$; $Q$ --- ортогональная матрица. По свойству ортогональных матриц:  $Q^{-1} = Q^\text{T}$. Рассмотрим ортогональное подобное преобразование :
	\begin{gather}
		B = Q A Q^{-1} = Q A Q^\text{T} \notag \\
		B^\text{T} = (Q A Q^\text{T})^\text{T} = (A Q^\text{T})^\text{T} Q^\text{T} = Q A^\text{T} Q^\text{T} = Q A Q^\text{T} = B \notag
	\end{gather}
	
	\item \textbf{-}
	\vspace*{0.2cm}
	
	\textit{\textbf{Ответ:}}
	
	\item \textbf{Почему на практике матрицу $A$ подобными преобразованиями вращения приводят только к форме Хессенберга, но не к треугольному виду?}
	
	\vspace*{0.2cm}
	
	\textit{\textbf{Ответ:}}
	\begin{enumerate}
		\item Могут <<потеряться>> комплексные собственные числа, поскольку матрицы поворота состоят из действительных чисел;
		\item Из предыдущего пункта следует, что нет явного способа привести матрицу к треугольному виду с использованием арифметики действительных чисел; 
		\item QR-алгоритм сохраняет форму Хессенберга на каждой итерации;
	\end{enumerate}

	
	\item \textbf{-}
	\vspace*{0.2cm}
	
	\textit{\textbf{Ответ:}}
	
	\item \textbf{Сойдется ли алгоритм обратных итераций, если в качестве начального приближения взять собственный вектор, соответствующий другому собственному значению? Что будет в этой ситуации в методе обратной итерации, использующем отношение Рэлея?}
	
	\vspace*{0.2cm}
	
	\textit{\textbf{Ответ:}}
	\begin{enumerate}
		\item Рассмотрим процесс метода обратных итераций:
		\[
		(A - \lambda^{*}_i E)y_{j+1} = x_j,
		\] 
		 где $\lambda^{*}_i$ приближение к собственному значению $\lambda_i$ и $x_{j} = \sfrac{y_j}{\norm{y_{j}}}$. Пусть $e_k$ собственный вектор не соответствующий собственному значению $\lambda_i$. Тогда:
		 \begin{gather}
		 (A - \lambda^{*}_i E) e_k = \lambda_k e_k - \lambda^*_i e_k = (\lambda_k - \lambda^*_i) e_k, \notag \\
		 \frac{1}{\lambda_k - \lambda^*_i}(A - \lambda^{*}_i E) e_k = (A - \lambda^{*}_i E)\frac{e_k}{\lambda_k - \lambda^*_i} =  e_k.
		 \end{gather}
		 Таким образом, если положить $x_0 = e_k$, тогда $x_1 = e_k$. Следовательно, итерационный процесс не сходится к собственному вектору, соответствующему значению  $\lambda_k$, но сходится за одну итерацию к собственному значению соответствующему вектору $e_k$.
		 
		 \item Рассмотрим процесс метода обратных итераций с соотношением Рэлея:
		 \[
		 (A - \lambda_k E)y_{j+1} = x_j,
		 \] 
		 где $\lambda_k = (A x_k, x_k)$ соотношение Рэлея и $x_{j} = \sfrac{y_j}{\norm{y_{j}}}$. Итерационный процесс должен сходиться к  собственному значению $\lambda$. Тогда, если положить $x_0 = e_k$, тогда матрица $(A - \lambda_k E)$ вырожденная, следовательно нет единственного решения, то есть итерационный процесс нельзя продолжить.
	\end{enumerate}
	
	\item \textbf{-}
	\vspace*{0.2cm}
	
	\textit{\textbf{Ответ:}}
	
	\item \textbf{ Предложите возможные варианты условий перехода к алгоритму со сдвигами. Предложите алгоритм выбора величины сдвига.}
	\vspace*{0.2cm}
	
	\textit{\textbf{Ответ:}}
	
	Сходимость QR алгоритма:
	\[
	\abs{a_{ij}^{(k)}} \le \abs{\dfrac{\lambda_i}{\lambda_j}}\cdot \abs{a_{ij}^{(k-1)}}, \;\;\; i > j, \;\;\; k = 1, 2, \ldots
	\]
	\begin{enumerate}
		\item
		Первый вариант: есть хорошее приближение собственного значения, тогда можно воспользоваться сдвигами, чтобы ускорить алгоритм; второй вариант: если матрица плохо обусловлена, то даже плохое приближение может ускорить процесс. В общем случае также следует применять сдвиги либо изначально, либо начиная с некоторой итерации, поскольку в процессе решения задачи возникают значения близкие к собственным, которые можно использовать как значение сдвига.  
		
		\item Можно воспользоваться теоремой Гершгорина, которая дает следующую оценку собственных значений:
		\[
		\abs{\lambda - a_{ii}} \le \sum\limits_{i \neq j}\abs{a_{ij}}.
		\] 
		Таким образом, при малой сумме элементов матрицы стоящих вне главной диагонали. 
	\end{enumerate}
	
	
	\item \textbf{-}
	\vspace*{0.2cm}
	
	\textit{\textbf{Ответ:}}
	
	\item \textbf{Приведите примеры использования собственных чисел и собственных векторов в численных методах}
	\vspace*{0.2cm}
	
	\textit{\textbf{Ответ:}}
	\begin{enumerate}
		\item Критерий сходимости стационарных итерационных методов решения СЛАУ (спектральный радиус меньше единицы); 
		\item Оптимальные итерационные параметры; для методов решения СЛАУ нередко требуют знания собственных значений матрицы (оптимальный параметр для метода простой итерации для симметричной, положительно определенной матрицы, для метода Ричардсона с Чебышевскими параметрами и т.д.);
		\item Оценка обусловленности матрицы;
		\item Решение системы линейных ОДУ с постоянными коэффициентами; 
		\item Привидение к канонической виду квадратичной формы. 
	\end{enumerate}
	
	\end{enumerate}
\end{document}