\documentclass[12pt, a4paper]{article}

\usepackage[utf8]{inputenc}
\usepackage[russian]{babel}
\usepackage{geometry}
\usepackage{mathtools}
\usepackage{verbatim}
\usepackage{indentfirst}
\usepackage{caption}
\usepackage{subcaption}
\usepackage{import}
\usepackage{xifthen}
\usepackage{pdfpages}
\usepackage{transparent}
\usepackage{graphicx}
\usepackage{caption}
\usepackage{hyperref}
\usepackage{float}

\newcommand{\norm}[1]{\lVert #1 \rVert}
\newcommand{\abs}[1]{\lvert #1 \rvert}
\usepackage[oglav,spisok,boldsect,eqwhole,figwhole,hyperref,hyperprint,remarks,greekit]{./style/fn2kursstyle}

\graphicspath{{./style/}{./figures/}}

\frenchspacing

\title{Решение задач интерполирования}
\lab{4}
\author{М.\,А.~Каган}
\creator{И.\,А.~Яковлев}
\supervisor{}
\group{ФН2-51Б}
\date{2024}

\begin{document}
	\maketitle
	\tableofcontents
	
	\newpage
	

	
	\section-{Контрольные вопросы}
	
	\begin{enumerate}
		\item \textbf{Можно ли использовать методы бисекции и Ньютона для нахождения кратных корней уравнения $f(x) = 0$, (т.е.тех, в которых одна или несколько первых производных функций $f(x)$ равны нулю)?}
		\vspace*{0.2cm}
		
		\textit{\textbf{Ответ:}}
		
		Принцип работы метода бисекции состоит в том, чтобы для функции $f(x)$ найти точку $x*$, в окрестности которой $f(x)$ меняет свой знак на противоположный. Таким образом, в случае четной кратности корней метод бисекции работать не сможет, так как не сможет локализовать место изменения знака функции. Метод Ньютона при этом может работать при любой кратности, так как его принцип заключается в последовательном приближении по касательным в окрестности искомой точки, на что кратность корня повлиять не сможет.
		
		\item \textbf{При каких условиях можно применять метод Ньютона для поиска корней уравнения $(5.1)$? При каких ограничениях на функцию $f(x)$ метод Ньютона обладает квадратичной скоростью сходимости? В каких случаях можно применять метод Ньютона для решения систем нелинейных уравнений?}
		\vspace*{0.2cm}
		
		\textit{\textbf{Ответ:}}

		Для применения метода Ньютона к уравнению $f(x) = 0$ достаточно того, чтобы функция $f(x)$ была непрерывно дифференцируема $f(x) \in C^1[a, b]$.

		Квадратичная сходимость появляется в случае, если первая производная функции $f(x)$ на отрезке $x \in [a, b]$ не находится в окрестности нуля $\abs{f'(x)} \geq m > 0$, при этом вторая производная на этом же отрезке должна быть ограничена $\abs{f''(x)} \leq M, \, m, \, \, M \in \mathbb{R}$. Можно заметить, что первое условие исключает возможность квадратичной сходимости для кратных корней.

		Для сходимости метода Ньютона при решении системы для функций $f_k(x), \, k = \overline{1, n}, \, x \in \mathbb{R}^n$ достаточно того, чтобы $f''_k(x) \in C^2[D]$, где $D$ --- область из $\mathbb{R}^n$, и матрица Якоби для любой точки из этой же области была невырождена.

		\item \textbf{Каким образом можно найти начальное приближение?}
		\vspace*{0.2cm}
		
		\textit{\textbf{Ответ:}}

		В случае, когда достоверно известно, что на некотором отрезке $[a, \, b]$ имеется корень, то для начального приближения можно взять или середину этого отрезка 
		\[
		x^{(0)} = \frac {a + b} 2,
		\]
		или место пересечения с осью $Ox$ линии, проведенной из вершины $(a, \, f(a))$ в вершину $(b, \, f(b))$
		\[
		x^{(0)} = \frac {f(a)b - f(b)a} {f(a) - f(b)}.
		\]

		\item \textbf{Можно ли использовать метод Ньютона для решения СЛАУ?}
		\vspace*{0.2cm}
		
		\textit{\textbf{Ответ:}}

		\item \textbf{Предложите альтернативный критерий окончания итерация в методе бисекции, в котором учитывалась бы возможность попадания очередного приближения в очень малую окресность корня уравнения.}
		\vspace*{0.2cm}
		
		\textit{\textbf{Ответ:}}

		Основной критерий останова, использующийся в методе бисекции, выполняется в случае, когда длина отрезка, в котором находится решение меньше $2 \varepsilon$:
		\[
		b^{(k+1)} - a^{(k+1)} < 2 \varepsilon.
		\]
		Так как решение уравнения ищется в соответствии с принципом вложенных отрезков (принципом Коши-Кантора), каждый раз ошибка между текущим и прошлым приближенным решением будет становиться меньше и приближаться к единственной точке, поэтому можно использовать критерий, оценивающий погрешность их разности:
		\[
		\abs{x^{(k+1)} - x^{(k)}} < \varepsilon.
		\]

		\item \textbf{Предложите различные варианты модификаций метода Ньютона. Укажите их достоинства и недостатки.}
		\vspace*{0.2cm}
		
		\textit{\textbf{Ответ:}}

		\item \textbf{Предложите алгоритм для исключения зацикливания метода Ньютона и выхода за пределы области поиска решения.}
		\vspace*{0.2cm}
		
		\textit{\textbf{Ответ:}}

		Зацикливание происходит в случае, если разность $\abs{x^{(k+1)} - x^{(j)}}$ для $j \geq 1$ мало изменяется с увеличением $k$. В этом случае можно попробовать внести погрешность в измерение, таким образом постепенно накапливая ошибку, позволяя разности изменяться сильнее.

		В случае, когда метод Ньютона выходит за пределы области поиска решения, можно перейти от метода касательных (Ньютона) к методу хорд. Так как такой метод соединяет две точки в области поиска решения, выхода за его пределы однозначно не произойдет. К тому же, переход к новому методу может позволять выйти из зацикливания.
	\end{enumerate}
	
	

	
	
\end{document}