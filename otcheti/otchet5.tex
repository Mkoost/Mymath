\documentclass[12pt, a4paper]{article}

\usepackage[utf8]{inputenc}
\usepackage[russian]{babel}
\usepackage{geometry}
\usepackage{mathtools}
\usepackage{verbatim}
\usepackage{indentfirst}
\usepackage{caption}
\usepackage{subcaption}
\usepackage{import}
\usepackage{xifthen}
\usepackage{pdfpages}
\usepackage{transparent}
\usepackage{graphicx}
\usepackage{caption}
\usepackage{hyperref}
\usepackage{float}

\newcommand{\norm}[1]{\lVert #1 \rVert}
\newcommand{\abs}[1]{\lvert #1 \rvert}
\usepackage[oglav,spisok,boldsect,eqwhole,figwhole,hyperref,hyperprint,remarks,greekit]{./style/fn2kursstyle}

\graphicspath{{./style/}{./figures/}}

\frenchspacing

\title{Решение задач интерполирования}
\lab{4}
\author{М.\,А.~Каган}
\creator{И.\,А.~Яковлев}
\supervisor{}
\group{ФН2-51Б}
\date{2024}

\begin{document}
	\maketitle
	\tableofcontents
	
	\newpage
	

	
	\section-{Контрольные вопросы}
	
	\begin{enumerate}
		\item \textbf{-}
		\vspace*{0.2cm}
		
		\textit{\textbf{Ответ:}}
		
		Чтобы итерационный процесс метода Ньютона для уравнения вида $f(x) = 0$ сходился, необходимо:
		\begin{enumerate}
			\item $f(x)$ должна иметь гладкую первую производную на $[a;b]$
		\end{enumerate}
		
		\item \textbf{Дайте определение константы Лебега. Оценка константы Лебега для различных сеток.}
		\vspace*{0.2cm}
		
		\textit{\textbf{Ответ:}}
	\end{enumerate}
	
	

	
	
\end{document}